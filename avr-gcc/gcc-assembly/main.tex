We show how to write a function in assembly and call it in a C program while programming the ATMega328P microcontroller in the Arduino.  This is done by controlling an LED. 
%
\begin{enumerate}[label=\arabic*.,ref=\theenumi]
\item Execute 
\begin{lstlisting}
cd avr-gcc/gcc-assembly/codes
make
\end{lstlisting}
\item Modify \textbf{main.c} and \textbf{Makefile} to turn the builtin led on.
\item Repeat the above exercise to turn the LED off.
\item Explain how the \textbf{disp\_led(0)} function is related to \textbf{Register R24} in \textbf{disp\_led} routine in \textbf{displedasm.S}.
	\\
\solution The function argument 0 in \textbf{disp\_led(0)} is passed on to R24 in the assembly routine for further operations.  Also, the registers R18-R24 are available for storing more function arguments according to the Table \ref{tab:avr-gcc/gcc-assembly/param_pass}.  More details are avilable in official ATMEL AT1886 reference.

\begin{table}[H]
\centering
	\input{avr-gcc/gcc-assembly/figs/param_pass.tex}
\caption{Relationship between Register in assembly and function argument in C}
\label{tab:avr-gcc/gcc-assembly/param_pass}
\end{table}

\item Write an assembly routine for controlling the seven segment display and call it in a C program.
\item Build a decade counter with \textbf{main.c} calling all functions from assembly routines.
\end{enumerate}



