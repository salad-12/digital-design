
We show how to interface an Arduino to a $16 \times 2$ LCD display using AVR-GCC.  This framework provides a useful platform for displaying the output of AVR-Assembly programs.
%
\begin{enumerate}[label=\arabic*.,ref=\theenumi]
	\item  The required components are listed in 
\tabref{tab:avr-gcc/lcd/components}
\begin{table}[H]
\input{avr-gcc/lcd/figs/components.tex}
\caption{}
\label{tab:avr-gcc/lcd/components}
\end{table}
\item Plug the LCD in Fig. \ref{fig:avr-gcc/lcd/lcd} to the breadboard.
\item Connect the Arduino pins to LCD pins as per Table \ref{table:lcd pins}.
%
\begin{figure}[H]
	\centering
\resizebox {\columnwidth} {!} {
\input{./avr-gcc/lcd/figs/lcd.tex}
}
\caption{LCD}
\label{fig:avr-gcc/lcd/lcd}
\end{figure}
%
\begin{table}[H]
	\centering
	\caption{Arduino to LCD Pin Connection.}             %\\	%
\input{avr-gcc/lcd/figs/table.tex}
\label{table:lcd pins}
\end{table}
\item Execute
%
\begin{lstlisting}
cd avr-gcc/lcd/codes
make
\end{lstlisting}
\item Modify the above code to display a string.
\item Modify the above code to obtain a decade counter so that the numbers from 0 to 9 are displayed on the lcd repeatedly.
\item Repeat the above exercies to display a string on the first line and a number on the second line of the lcd.
\item Write assembly routines for driving the lcd.
\end{enumerate}
%


