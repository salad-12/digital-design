We explain Karnaugh maps (K-map) by finding the
logic functions for the incrementing decoder

\begin{enumerate}[label=\arabic*.,ref=\theenumi]
%
	\item The incrementing decoder   takes the numbers $0,,\dots,9$ in binary as inputs and generates
the consecutive number as output.  The corresponding truth table is available in \tabref{tab:ide/7447/counter_decoder}
%\input{ide/kmap/figs/counter_decoder}

%
\item Using Boolean logic, output $A$  in \tabref{tab:ide/7447/counter_decoder} can be expressed in terms of the inputs $W,X,Y,Z$ as
\begin{multline}
\label{eq:A}
A = W^{\prime}X^{\prime}Y^{\prime}Z^{\prime} + W^{\prime}XY^{\prime}Z^{\prime}
+W^{\prime}X^{\prime}YZ^{\prime}
\\
+W^{\prime}XYZ^{\prime}
+W^{\prime}X^{\prime}Y^{\prime}Z
\end{multline}
\item K-Map for $A$: 
The expression in \eqref{eq:A}  can be minimized using the K-map in Fig \ref{fig:kmap_A}
In Fig \ref{fig:kmap_A},  the {\em implicants} in boxes 0,2,4,6 result in $W^{\prime}Z^{\prime}$  The implicants in
boxes 0,8 result in $W^{\prime}X^{\prime}Y^{\prime}$  Thus, after minimization using Fig \ref{eq:kmap_A},  \eqref{eq:A} can be expressed as
%
\begin{equation}
\label{eq:kmap_A}
A = W^{\prime}Z^{\prime}+W^{\prime}X^{\prime}Y^{\prime}
\end{equation}
%
Using the fact that
\begin{align}
\label{eq:boolean}
\begin{split}
X+X^{\prime} &= 
\\
XX^{\prime} &= 0,
\end{split}
\end{align}
%
derive \eqref{eq:kmap_A} from \eqref{eq:A} algebraically
%
%
%
\begin{figure}[!ht]
\resizebox {\columnwidth} {!} {
\input{ide/kmap/figs/kmap_A}
}
\caption{K-map for $A$}
\label{fig:kmap_A}
\end{figure}
%
\item K-Map for $B$:
From Table \ref{tab:ide/7447/counter_decoder}, using boolean logic,
\begin{equation}
\label{eq:B}
B = WX^{\prime}Y^{\prime}Z^{\prime} + W^{\prime}XY^{\prime}Z^{\prime}
+WX^{\prime}YZ^{\prime}
+W^{\prime}XYZ^{\prime}
\end{equation}
%
\begin{figure}[!ht]
\resizebox {\columnwidth} {!} {
\input{ide/kmap/figs/kmap_B}
}
\caption{K-map for $B$}
\label{fig:kmap_B}
\end{figure}
%
Show that \eqref{eq:B} can be reduced to
\begin{equation}
\label{eq:kmap_B}
B = WX^{\prime}Z^{\prime} + W^{\prime}XZ^{\prime}
\end{equation}
using Fig \ref{fig:kmap_B}
\item Derive \eqref{eq:kmap_B} from \eqref{eq:B} algebraically using \eqref{eq:boolean}
%
%
\item {K-Map for $C$: }
From Table \ref{tab:ide/7447/counter_decoder}, using boolean logic,
\begin{equation}
\label{eq:C}
C = WXY^{\prime}Z^{\prime} + W^{\prime}X^{\prime}YZ^{\prime}
+WX^{\prime}YZ^{\prime}
+W^{\prime}XYZ^{\prime}
\end{equation}
%
%
\begin{figure}[!ht]
\resizebox {\columnwidth} {!} {
\input{ide/kmap/figs/kmap_C}
}
\caption{K-map for $C$}
\label{fig:kmap_C}
\end{figure}
%
Show that \eqref{eq:C} can be reduced to
\begin{equation}
\label{eq:kmap_C}
C = WXY^{\prime}Z^{\prime}  +  X^{\prime}YZ^{\prime} + W^{\prime}YZ^{\prime}
\end{equation}
using Fig \ref{fig:kmap_C}
%
\item 
Derive \eqref{eq:kmap_C} from \eqref{eq:C} algebraically using \eqref{eq:boolean}
%
\item {K-Map for $D$: }
From Table \ref{tab:ide/7447/counter_decoder}, using boolean logic,
\begin{equation}
\label{eq:D}
D = WXYZ^{\prime} + W^{\prime}X^{\prime}Y^{\prime}Z
\end{equation}
%
\begin{figure}[!ht]
\resizebox {\columnwidth} {!} {
\input{ide/kmap/figs/kmap_D}
}
\caption{K-map for $D$}
\label{fig:kmap_D}
\end{figure}
%
\item 
Minimize \eqref{eq:D} using Fig \ref{fig:kmap_D}
%
\item Execute the code in
\begin{lstlisting}
ide/7447/codes/inc_dec/inc_dec.cpp
\end{lstlisting}
%
and modify it using the K-Map equations for A,B,C and D Execute and verify
\item {Display Decoder:}
Table \ref{table:disp_dec} is the truth table for the display decoder in Fig.
%\ref{fig:dec_counter}  
\ref{fig:7447}.
Use K-maps to obtain the minimized expressions for $a,b,c,d,e,f,g$ in terms of $A,B,C,D$ with and without don't care conditions
%
\begin{table*}[!ht]
	\centering
\input{ide/kmap/figs/disp_dec}
\caption{Truth table for display decoder.}
\label{table:disp_dec}
\end{table*}
\end{enumerate}

