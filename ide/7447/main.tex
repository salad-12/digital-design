Here we show how to use the 7447 BCD-Seven Segment Display decoder to learn Boolean logic.
\subsection{Hardware}
\begin{enumerate}[label=\arabic*.,ref=\theenumi]
	%	\numberwithin{figure}{section}
\item
Make connections between the seven segment display in Fig. \ref{fig:sevenseg} and the  7447 IC in Fig. \ref{fig:7447} as shown in Table \ref{table:7447_disp}

\begin{table}[!ht]
\centering
\input{ide/7447/figs/components.tex}
\caption{7447 components}
\label{table:components-7447}
\end{table}
%
\begin{table}[!ht]
\centering
\input{ide/7447/figs/7447_disp.tex}
\caption{}
\label{table:7447_disp}
\end{table}
%
\iffalse
\begin{figure}[!ht]
\begin{center}
\resizebox {0.5\columnwidth} {!} {
\input{ide/7447/figs/sevenseg.tex}
}
\end{center}
\caption{}
\label{fig:sevenseg}
\end{figure}
\fi
\item
Make connections to the lower pins of the 7447 according to
Table \ref{table:bin2dec} and connect $V_{CC} = 5$V. You should see the number 0 displayed for 0000 and 1 for 0001.

%
\begin{table}[!ht]
\centering
\input{ide/7447/figs/bin2dec.tex}
\caption{}
\label{table:bin2dec}
\end{table}
%
\begin{figure}[!ht]
\begin{center}
\resizebox {\columnwidth} {!} {
\input{ide/7447/figs/7447.tex}
}
\end{center}
\caption{}
\label{fig:7447}
\end{figure}
%
\item
Complete Table \ref{table:bin2dec} by generating all numbers between 0-9.

	\end{enumerate}
\subsection{Software}
\begin{enumerate}[label=\arabic*.,ref=\theenumi]
		\numberwithin{figure}{section}
\item
Now make the connections as per Table \ref{table:7447_ard}  and execute the following program 
\begin{lstlisting}
ide/7447/codes/gvv_ard_7447/gvv_ard_7447.cpp
\end{lstlisting}

\begin{table}[!ht]
\centering
\input{ide/7447/figs/7447_ard.tex}
\caption{}
\label{table:7447_ard}
\end{table}
In the  truth table in Table \ref{tab:ide/7447/counter_decoder},  $W,X,Y,Z$ are the inputs
and $A,B,C,D$ are the outputs. This table represents the system that increments the numbers 0-8 by 1 and resets the number 9 to 0
%
Note that  $D = 1$ for the inputs $0111$ and $1000$.  Using {\em boolean} logic,
%
\begin{equation}
\label{bool_logic}
D = WXYZ^{'} + W^{'}X^{'}Y^{'}Z
\end{equation}
%
Note that $0111$ results in the expression $WXYZ^{'}$ and $1000$ yields $W^{'}X^{'}Y^{'}Z$. 
%
\item
The code below realizes the Boolean logic for B, C and D in  Table \ref{tab:ide/7447/counter_decoder}.  Write the logic for A and verify.
\begin{lstlisting}
ide/7447/codes/inc_dec/inc_dec.ino
\end{lstlisting}

%		\begin{table}[!t]
\begin{table*}[!t]
\centering
%\resizebox{\columnwidth}{!}{ 
\input{ide/7447/figs/counter_decoder.tex}
%}
\caption{Truth table for incrementing Decoder.}
\label{tab:ide/7447/counter_decoder}
\end{table*}
%\end{table}
\item
Now make additional connections as shown in Table \ref{table:ip_7447_ard} and execute the following code.  Comment.
%			\lstinputlisting{ide/7447/codes/ip_inc_dec/ip_inc_dec.ino}
\begin{lstlisting}
ide/7447/codes/ip_inc_dec/ip_inc_dec.cpp
\end{lstlisting}

\solution
In this exercise, we are taking the number 5 as input to the arduino and displaying it on the seven segment display using the 7447 IC.
\begin{table}[!ht]
\centering
\input{ide/7447/figs/ip_7447_ard.tex}
\caption{}
\label{table:ip_7447_ard}
\end{table}
\item
Verify the above code for all inputs from 0-9.

\item
Now write a program where 
\begin{enumerate}
\item the binary inputs are given by
connecting to 0 and 1 on the breadboard
\item incremented by 1 using Table \ref{tab:ide/7447/counter_decoder} and
\item the incremented value is displayed on the seven segment display.
\end{enumerate}

\item
Write the truth table for the 7447 IC and obtain the corresponding boolean logic equations. 

\item
Implement the 7447 logic in the arudino.  Verify that your arduino now behaves like the 7447 IC.
	\end{enumerate}
