\begin{enumerate}[label=\arabic*.,ref=\theenumi]
\item 	The state diagram of a sequence detector is shown in
  \figref{fig:gate/ec/2020/39/1}.
		 State $S_0$ is the initial state of the sequence detector. If the output is 1, then
\hfill (GATE EC 2020)
	\begin{figure}[H]
    \centering
    \resizebox{\columnwidth}{!}{%
  \input{gate/ec/2020/39/figs/diagram.tex}
		}
  \caption{}
  \label{fig:gate/ec/2020/39/1}		
  \end{figure}	 
\begin{enumerate}
 \item the sequence 01010 is detected
 \item the sequence 01011 is detected
 \item the sequence 01110 is detected
 \item the sequence 01001 is detected	 
\end{enumerate}	
\item A sequence detector is designed to detect precisely 3 digital inputs, with overlapping sequences detectable. For the sequence $(1,0,1)$ and input data $(1,1,0,1,0,0,1,1,0,1,0,1,1,0)$, what is the output of this detector?
		\begin{enumerate}
			\item 1,1,0,0,0,0,1,1,0,1,0,0
			\item 0,1,0,0,0,0,0,1,0,1,0,0
			\item 0,1,0,0,0,0,0,1,0,1,1,0
			\item 0,1,0,0,0,0,0,0,1,0,0,0
		\end{enumerate}
		\hfill (GATE EE 2020)
\item Consider a $3$-bit counter, designed using T flip-flops, as shown below
in \figref{fig:3bitcounter.jpg}
     \begin{figure}[H]
\centering
\includegraphics[width=0.75\columnwidth]{ide/fsm/figs/3bitcounter.jpg}
\caption{}
\label{fig:3bitcounter.jpg}
\end{figure}
Assuming the initial state of the counter given by $PQR$ as $000$,what are the next three states?
                 \hfill(GATE-CS2021)
\begin{enumerate}[label=(\Alph*)]
\item $011, 101, 000$
\item $010, 101, 000$
\item $010, 101, 000$
\item $010, 101, 000$
\end{enumerate}

\item The state diagram of a sequence detector is shown below. state S0 is the initial state of the sequence detector. If the output is 1,then
                                        \hfill(GATE-EC2020,39)
	\begin{figure}[H]
    \centering
    \resizebox{\columnwidth}{!}{%
    \input{ide/fsm/figs/fig6.tex}
	}
    \caption{}
	\label{fig:GATE EC2020,39}
\end{figure}
%    	
\begin{enumerate}
\item   the sequence $01010$ is detected.
\item   the sequence $01011$ is detected.
\item   the sequence $01110$ is detected.
\item   the sequence $01001$ is detected.
\end{enumerate}
%
\item The state transition diagram for the circuit shown in 
	\figref{fig:GATE IN2019,39}
	is
                         \hfill(GATE-IN2019,39)
\begin{figure}[H]
\centering
    \resizebox{\columnwidth}{!}{%
\input{ide/fsm/figs/fig11.tex}
	}
    \caption{}
	\label{fig:GATE IN2019,39}
\end{figure}

\begin{enumerate}
%% options1
\item 
	\figref{fig:GATE IN2019,39-1}
\begin{figure}[H]
\centering
    \resizebox{\columnwidth}{!}{%
\input{ide/fsm/figs/opt1.tex}
	}
    \caption{}
	\label{fig:GATE IN2019,39-1}
\end{figure}
%% options2
\item 
	\figref{fig:GATE IN2019,39-2}
\begin{figure}[H]
\centering
    \resizebox{\columnwidth}{!}{%
\input{ide/fsm/figs/opt2.tex}
	}
    \caption{}
	\label{fig:GATE IN2019,39-2}
\end{figure}
\item 
	\figref{fig:GATE IN2019,39-3}
\begin{figure}[H]
\centering
    \resizebox{\columnwidth}{!}{%
\input{ide/fsm/figs/opt3.tex}
	}
    \caption{}
	\label{fig:GATE IN2019,39-3}
\end{figure}
\item 
	\figref{fig:GATE IN2019,39-4}
\begin{figure}[H]
\centering
    \resizebox{\columnwidth}{!}{%
\input{ide/fsm/figs/opt4.tex}
	}
    \caption{}
	\label{fig:GATE IN2019,39-4}
\end{figure}

\end{enumerate}

\item A sequence detector is designed to detect precisely $3$ digital inputs, with overlapping sequences detectable. For the sequence $\brak{1,0,1}$ and input data $\brak{1,1,0,1,0,0,1,1,0,1,0,1,1,0}$, the output sequence is
$\hfill\brak{GATE\enspace EE2020-15}$
   \begin{enumerate}
  \item  $\brak{1,1,0,0,0,0,1,1,0,1,0,0}$
  \item $\brak{0,1,0,0,0,0,0,1,0,1,0,0}$
  \item $\brak{0,1,0,0,0,0,0,1,0,1,1,0}$
  \item $\brak{0,1,0,0,0,0,0,0,1,0,0,0}$

\end{enumerate}

\item A finite state machine (FSM) is implemented using the D flip-flops $A$ and $B$, and logic gates, as shown in  
\figref{fig:ide/fsm/figs/circuit}
	below. The four possible states of the FSM are $Q_AQ_B = 00, 01, 10$ and	 $11$.  
%
\begin{figure}[H]
\centering
\resizebox{\columnwidth}{!}{%
	\input{ide/fsm/figs/circuit.tex}
}%
	\caption{}
\label{fig:ide/fsm/figs/circuit}
\end{figure}
Assume that $X_{IN}$ is held at a constant logic level throughout the operation of the FSM. When the FSM is initialized to the state $Q_AQ_B = 00$ and clocked, after a few clock cycles, it starts cycling through
\begin{enumerate}
\item all of the four possible states if $X_{IN} = 1$
\item three of the four possible states if $X_{IN} = 0$
\item only two of the four possible states if $X_{IN} = 1$
\item only two of the four possible states if $X_{IN} = 0$
\end{enumerate}
\hfill{(GATE EC 2017)}

\end{enumerate}
