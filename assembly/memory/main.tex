This manual shows how to use the Atmega328p internal memory for a decade counter through a loop.
\begin{enumerate}[label=\arabic*.,ref=\theenumi]
\item Exectute the following code by connecting the Arduino to 7447 through pins 2,3,4,5. The seven segment display should be connected to 7447.
\begin{lstlisting}
assembly/memory/codes/mem.asm
\end{lstlisting}
\item Explain the following instructions
\begin{lstlisting}
ldi xl,0x00
ldi xh,0x01
ldi r16,0b00000000
st x,r16
\end{lstlisting}
\solution X=R27:R26, Y=R29:R28, and Z=R31:R30 where R27:R26 represents XH:XL. The above instructions load 0b00000000 into
the memory location X=0x0100.
\item What does the \textbf{loop\_cnt} routine do?
\begin{lstlisting}
ldi r16,0b00000000
ldi r17,0x09
loop_cnt:
inc r16
inc xl
st x,r16
dec r17
brne loop_cnt
\end{lstlisting}
\solution The routine loads the numbers 1-9 in memory locations 0x0101 - 0x0109.

\item Revise your code by using a timer for giving the delay.
\end{enumerate}


